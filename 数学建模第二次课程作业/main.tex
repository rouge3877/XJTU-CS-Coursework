% 看视频,学习\textbf{PageRank}排序算法(群文件6.1-6.5) 并回答以下问题:
% 1、创建谷歌时,拉里·佩奇和谢尔·盖布林对互联网的基本认识是什么?
% 2、谷歌\textbf{PageRank}模型的核心思想是什么?
% 3、请写出网页的重要性的计算公式
% 4、互联网规模巨大,但思考和分析这个问题,我们却在一个规模非常小的问题上花了大量时间,还称之为“武林秘笈”。请谈谈你的看法?

% 作业要求:
%  (1)word版本;小四号字体,单倍行距,所有的公式、符号都要通过公式编辑器键入。
%   (2)内容包括:题目,假设,符号说明,解答、结论等。
%   (3)若认为自己做的不错,可制作PPT课堂宣讲(有加分)。
% 作业提交:---------第1-4次作业全部写在一个文档中,以 “学号+姓名”命名,在思源学堂提交.


\documentclass[a4,10pt,zihao=-4]{ctexart}
\linespread{1.0} % 设置单倍行距

\usepackage{ctex}
\usepackage[utf8]{inputenc}
\usepackage{amsfonts,amsmath,amscd,amssymb,amsthm}
\usepackage{latexsym,bm}
\usepackage{cite}
\usepackage{mathtools,mathdots,graphicx,array}
\usepackage{fancyhdr}
\usepackage{lastpage}
\usepackage{color}
\usepackage{enumitem}
\usepackage{mpdoc}
\usepackage{diagbox}
\usepackage{xcolor,tcolorbox,tikz,tkz-tab,mdframed,tikz-cd}
\usepackage{framed}
\usepackage{verbatim}
\usepackage{extarrows}
\usepackage{pythonhighlight}
\usepackage{fontspec}

\newcommand*{\dif}{\mathop{}\!\mathrm{d}}
\newcommand*{\arsinh}{\mathop{}\!\mathrm{arsinh}}
\newcommand*{\artanh}{\mathop{}\!\mathrm{artanh}}
\newcommand*{\arcosh}{\mathop{}\!\mathrm{arcosh}}
\newcommand*{\Li}{\mathop{}\!\textrm{Li}}



\begin{document}
\pagenumbering{roman}
\title{数学建模:第5 - 6次作业}

\author{Name STUID Class}
\date{2024年3月}
\maketitle
\tableofcontents
\newpage
\pagenumbering{arabic}
\newpage

%---------------------------------------------------------------------------------
\section{第五次作业:\textbf{PageRank}算法及相关}
\subsection{问题一:拉里·佩奇和谢尔·盖布林对互联网的基本认识}
\subsubsection{问题}
创建谷歌时,拉里·佩奇和谢尔·盖布林对互联网的基本认识是什么?

\subsubsection{解答}

拉里·佩奇和谢尔·盖布林在创建谷歌之初,对互联网的基本认识可以概括为以下几点:

\begin{itemize}
    \item 搜索的必要性: 由于互联网上信息的分散性,他们认识到需要一种高效的方式来帮助用户查找他们所需的信息,因此决定开发一种更好的搜索方式。
    \item 信息的无限性和分散性: 他们意识到互联网上存在着大量的信息,但这些信息通常是分散的,难以有效地获取和组织。
    % \item 用户体验的重要性: 他们关注用户的体验,意识到任何成功的搜索引擎都必须提供简单易用的界面,使用户能够轻松地找到所需的信息。
    % \item 技术创新的价值: 他们相信技术创新能够改善人们的生活,并且致力于通过创造性的技术解决方案来实现这一目标,包括改进搜索算法和用户体验。
\end{itemize}

\subsection{问题二:\textbf{PageRank}模型的核心思想}
\subsubsection{问题}
谷歌\textbf{PageRank}模型的核心思想是什么?

\subsubsection{解答}
\textbf{PageRank}模型的核心思想在于通过\textbf{分析网页之间的链接结构来确定网页的重要性}。具体而言,该模型假设用户通过点击网页上的超链接来浏览其他相关网页,并认为用户跳转到某个网页的概率与该网页的重要性有关。因此,\textbf{PageRank}算法将网页的重要性建模为一个基于链接结构的迭代计算过程,通过不断更新网页的\textbf{PageRank}值来反映其在网络中的相对重要性。

简单来说,\textbf{PageRank} 是一个定义在整个网页集合上的函数,为每个网页分配一个正实数,代表该网页的重要性。这些数值组成一个向量,其中较高的 \textbf{PageRank} 值意味着该网页在重要性上的优势,因此在搜索结果中可能会被优先显示。

可以将整个互联网视为一个巨大的有向图,其中每个网页是一个节点,而超链接则是从一个页面指向另一个页面的有向边。基于这个视角,我们可以构建一个随机游走模型,也就是一阶马尔可夫链。在这个模型中,我们假设一个虚拟的网页浏览者会随机地、按照等概率地跟随一个页面上的任何一个超链接到另一个页面,并持续这种随机跳转。在长时间内,这种随机跳转的行为会形成一个稳定的模式,即马尔可夫链的平稳分布。每个网页的 \textbf{PageRank} 值,实际上就是在这个平稳分布中的概率。

\subsection{问题三:网页的重要性的计算公式}
\subsubsection{问题}
请写出网页的重要性的计算公式

\subsubsection{解答}
给定一个包含 n 个结点的强连通且非周期性的有向图,在其基础上定义随机游走模型。假设转移矩阵为 $M$ , 在时刻$ 0,1,2, \cdots ,t, \cdots $访问各个结点的概率分布为

$$
{R_0},M{R_0},{M^2}{R_0}, \cdots ,{M^t}{R_0}, \cdots
$$

则极限 $ \lim \limits_{t \to \infty } {M^t}{R_0} = R$ 存在,极限向量 $R$ 表示马尔可夫链的平稳分布,满足

$$
MR=R
$$

给定一个包含 $n$ 个结点 ${v_1},{v_2}, \cdots ,{v_n}$的强连通且非周期性的有向图,在有向图上定义随机游走模型,即一阶马尔可夫链。随机游走的特点是从一个结点到有有向边连出的所有结点的转移概率相等,转移矩阵为 $M$ 。这个马尔可夫链具有平稳分布 $R$。其中平稳分布 $R$ 称为这个有向图的 \textbf{PageRank}。 $R$ 的各个分量称为各个结点的 \textbf{PageRank} 值。

$$
R=\left[\begin{array}{c} P R\left(v_{1}\right) \\ P R\left(v_{2}\right) \\ \vdots \\ P R\left(v_{n}\right) \end{array}\right]\\
$$


其中 $PR(V_i) , i = 1,2, \cdots ,n$ ,表示结点的的 \textbf{PageRank} 值。

显然有

$$
P R\left(v_{i}\right) \geqslant 0, \quad i=1,2, \cdots, n 
$$

$$
\sum_{i=1}^{n} P R\left(v_{i}\right)=1 
$$


$$
P R\left(v_{i}\right)=\sum_{v_{j} \in M\left(v_{i}\right)} \frac{P R\left(v_{j}\right)}{L\left(v_{j}\right)}, \quad i=1,2, \cdots, n
$$

这里 $M(v_i)$ 表示指向结点 $v_i$ 的结点集合,$ L(v_j)$ 表示结点 $v_j$ 连出的有向边的个数。\textbf{PageRank} 的基本定义是理想化的情况,在这种情况下,\textbf{PageRank} 存在,而且可以通过不断迭代求得\textbf{PageRank}值。

\subsubsection{补充解答}

网页的重要性通常使用\textbf{PageRank}算法进行计算,\textbf{完整地来讲},其计算公式应该如下面所示:
\[ PR_i = \frac{1 - d}{N} + d \sum_{j \in M_i} \frac{PR_j}{L_j} \]

其中:

\begin{itemize}
    \item \( PR_i \) 是网页 \( i \) 的\textbf{PageRank}值
    \item \( d \) 是一个称为阻尼因子的常数,通常取值为0.85,用于控制随机跳转到其他网页的概率。
    \item \( N \) 是网页总数。
    \item \( M_i \) 是指向网页 \( i \) 的网页集合。
    \item \( L_j \) 是网页 \( j \) 的出链数量,即指向其他网页的链接数。
\end{itemize}

该公式表明,网页 \( i \) 的\textbf{PageRank}值由两部分组成,一部分是平均分配给所有网页的初始值 \( \frac{1 - d}{N} \),表示用户在随机情况下跳转到网页 \( i \) 的概率。另一部分是通过网页 \( i \) 的入链贡献的\textbf{PageRank}值,根据指向网页 \( i \) 的网页的\textbf{PageRank}值和这些网页的出链数量进行加权求和。

\subsection{问题四:在小规模问题上思考和分析的意义}
\subsubsection{问题}
互联网规模巨大,但思考和分析这个问题,我们却在一个规模非常小的问题上花了大量时间,还称之为“武林秘笈”。请谈谈你的看法?

\subsubsection{解答}

这种现象确实存在,并且在技术和科学领域中很常见。将大问题简化为小问题,并在小规模示例上进行深入思考和分析,尽管这些问题可能规模较小,甚至与实际问题有一定的脱节,但通过解决这些问题,我们可以更好地理解基本原理、算法和方法,并为解决更大规模的问题奠定基础。

一个重要的观点,即使数学模型简单,也能够解决复杂的实际问题。关键在于\textbf{对问题的深刻理解以及将数学知识与实际问题相结合}。举例来说,考虑一个简单的人口增长模型。假设我们想要预测未来五年内某城市的人口增长情况。尽管这个问题涉及到复杂的人口迁移、出生率、死亡率等因素,但我们可以使用一个简单的指数增长模型来描述人口增长的趋势。这个模型可能只涉及到简单的指数函数和一些基本的统计数据,但通过适当地调整模型参数,我们仍然可以得出对未来人口增长的合理预测。另一个重点是,一个好的数学模型并不一定需要复杂深奥的数学知识。这意味着在建立数学模型时,我们应该注重的是将数学工具与解决的实际问题相结合,而\textbf{不是盲目地追求数学的复杂性}。例如,在金融领域,为了预测股票价格的走势,我们可以使用简单的时间序列分析方法,如移动平均模型或指数平滑模型。尽管这些模型可能不涉及高阶的数学理论,但它们仍然可以提供对股票价格趋势的有用预测。

从小规模问题开始,并逐步扩展到大规模实际问题的重要性。这种渐进式的方法可以帮助我们逐步理解问题的复杂性,并且在解决实际问题时更容易应用和调整数学模型。例如,在城市交通规划中,我们可以从一个小区域开始,使用简单的交通流模型来评估道路拥堵情况,然后逐步扩展到整个城市的交通网络,考虑更多的交通流量和路线优化问题。

所以说,重要的是对问题进行深刻理解,并将数学知识与实际问题相结合,而不是盲目追求数学的复杂性。通过简单而有效的数学模型,以及渐进式的解决方法,我们可以解决各种复杂的实际问题。




%---------------------------------------------------------------------------------
\newpage
\section{第七章研究课题3 - 利润最大的精炼计划}
\subsection{题目}
加工一种食用油需要精炼若干种原料油并把它们混合起来。原料油的来源有两类共5种:植物油VEG1、植物油VEG2,非植物油OIL1、非植物油OIL2、非植物油OL3。购买每种原料油的价格(英镑/t)如 表\ref{料油价格} 所示,最终产品以150英镑/t的价格出售。植物油和非植物油需要在不同的生产线上进行精炼。每月能够精炼的植物油不超过 200 t,非植物油不超过 250t。在精炼过程中,重量没有损失,精炼费用可忽略不计。最终产品要符合硬度的技术条件。按照硬度计量单位,它必须在 3 - 6 范围内。假定硬度的混合是线性的,而原材料的硬度如 表\ref{原料油硬度表} 所示。为使利润最大,应该怎样指定它的月采购量和加工计划。(注:1英磅约等于8.8652 元人民币)

\begin{table}[!ht]
    \centering
    \caption{料油价格}
    \begin{tabular}{|c|c|c|c|c|c|}
    \hline
        原料油 & VEG1 & VEG2 & OIL1 & OIL2 & OIL3 \\ \hline
        价格/(英镑/t) & 110 & 120 & 130 & 110 & 115 \\ \hline
    \end{tabular}
    \label{料油价格}
\end{table}

\begin{table}[!ht]
    \centering
    \caption{原料油硬度表}
    \begin{tabular}{|c|c|c|c|c|c|}
    \hline
        原料油 & VEG1 & VEG2 & OIL1 & OIL2 & OIL3 \\ \hline
        硬度值/($\text{mol} \cdot \text{L}^{-1}$) & 8.8 & 6.1 & 2.0 & 4.2 & 5.0 \\ \hline
    \end{tabular}
    \label{原料油硬度表}
\end{table}

\subsection{问题分析}
此题为有约束的线性规划问题。需要通过数学中关于线性规划的方法来确定每种原料油的购买量以及最佳的加工计划,以实现最大化利润的目标。我们需要考虑约束条件,如每月精炼的限制和最终产品的硬度要求,以及利润最大化的目标函数。通过合理选择原料油的比例,我们可以确保最终产品满足技术要求,并在销售中获得最大的利润。

\subsection{合理假设}
\begin{itemize}
    \item \textbf{线性混合假设}:问题中假设最终产品的硬度是原材料硬度的线性组合,即硬度的混合是线性的。这意味着最终产品的硬度可以通过各种原料的硬度按比例加权平均来计算。这种假设在实际情况下可能是近似成立的,但并不总是准确,特别是在不同种类的油混合时可能会存在非线性效应。

    \item \textbf{精炼过程中的无损假设}:问题中提到,在精炼过程中,重量没有损失。这意味着原料油的购买量与最终产品的产量是直接相关的,没有额外的损耗。在实际情况中,可能会有一些损耗,例如在加工过程中可能会有部分原料油残留在设备中或发生蒸发损失。

    \item \textbf{加工成本忽略假设}:问题中没有考虑加工成本,即假设加工过程中的成本可以忽略不计。在实际情况中,加工成本可能包括能源消耗、人工成本、设备折旧等,这些成本可能会影响最终产品的利润。

    \item \textbf{销售价格不变假设}:问题中假设最终产品的销售价格是固定的,即每吨150英镑。在实际情况中,市场价格可能会波动,这可能会影响最终产品的销售收入和利润。
\end{itemize}

\subsection{符号说明}
首先,我们需要定义一些变量:
\begin{itemize}
    \item \( x_1, x_2, x_3, x_4, x_5 \):分别表示购买 VEG1、VEG2、OIL1、OIL2、OIL3 的数量(单位:t)。
    \item \( y \):表示最终产品的销售量(单位:t)。
\end{itemize}

\subsection{模型建立}

为了确定最佳的采购量和加工计划,我们可以使用线性规划方法。我们的目标是最大化利润,利润可以通过销售收入减去成本来计算。成本是原材料的购买成本和加工成本,其中加工成本可以忽略不计。销售收入取决于最终产品的销售量,销售量取决于原材料的混合比例和硬度要求。

下面,我们确定一些约束条件:

1. 原材料购买的约束:每种原材料的购买量不能为负,且不能超过月产能。
 
\begin{align*}
&0 \leq x_1, x_2, x_3, x_4, x_5 \leq \infty \\
&x_1 + x_2 \leq 200 \\
&x_3 + x_4 + x_5 \leq 250 \\
\end{align*}
 

2. 硬度要求的约束:最终产品的硬度必须在 3 到 6 之间。
\[
3 \leq \frac{8.8x_1 + 6.1x_2 + 2.0x_3 + 4.2x_4 + 5.0x_5}{x_1 + x_2 + x_3 + x_4 + x_5} \leq 6
\]

接下来,我们可以设置目标函数,即最大化利润。利润可以表示为销售收入减去购买成本:

\[
\text{Profit} = 150y - (110x_1 + 120x_2 + 130x_3 + 110x_4 + 115x_5)
\]


\subsection{模型求解}
现在,我们可以将这些信息整理成线性规划问题的标准形式,并写程序求解。

\[ 
\textbf{Maximize: Profit} = 150y - (110x_1 + 120x_2 + 130x_3 + 110x_4 + 115x_5) 
\]


\begin{align*}
\textbf{Subject to:\,\,\,\,\,\,\,\,\,\,\,\,\,\,}
&0 \leq x_1, x_2, x_3, x_4, x_5  \leq \infty \\ 
&x_1 + x_2 \leq 200 \\
&x_3 + x_4 + x_5 \leq 250 \\
&3 \leq \frac{8.8x_1 + 6.1x_2 + 2.0x_3 + 4.2x_4 + 5.0x_5}{x_1 + x_2 + x_3 + x_4 + x_5} \leq 6 \\
& y = x_1 + x_2 + x_3 + x_4 + x_5
\end{align*}


使用python中的线性规划库来解决这个问题,以确定最佳的采购量和加工计划,以最大化利润。下面是解决问题部分的python代码:


% Maximize: Profit = 150y - (110x_1 + 120x_2 + 130x_3 + 110x_4 + 115x_5) 

% Subject to
% 0 <= x_1, x_2, x_3, x_4, x_5 
% x_1 + x_2 <= 200
% x_3 + x_4 + x_5 <= 250
% 3 <= (8.8x_1 + 6.1x_2 + 2.0x_3 + 4.2x_4 + 5.0x_5)/(x_1 + x_2 + x_3 + x_4 + x_5) <= 6




\begin{python}
from scipy.optimize import linprog

# 定义目标函数的系数
c = [-110, -120, -130, -110, -115]

# 定义不等式约束的系数
A = [[1, 1, 0, 0, 0], [0, 0, 1, 1, 1], [-8.8, -6.1, -2.0, -4.2, -5.0]]
b = [200, 250, -3]

# 定义变量的边界
x0_bounds = (0, None)
x1_bounds = (0, None)
x2_bounds = (0, None)
x3_bounds = (0, None)
x4_bounds = (0, None)

# 调用线性规划求解器
res = linprog(c, A_ub=A, b_ub=b, bounds=[x0_bounds, x1_bounds, x2_bounds, x3_bounds, x4_bounds])

# 输出结果
print("最大利润:", -res.fun)
print("最优解:")
print("x1 =", res.x[0])
print("x2 =", res.x[1])
print("x3 =", res.x[2])
print("x4 =", res.x[3])
print("x5 =", res.x[4])

\end{python}


\subsection{结论}
求解得到最大利润为 $56499.99$英镑,且在最优解中:
\begin{itemize}
    \item $x1 = 1.0039\times 10^{-6}$
    \item $x2 = 199.9999$
    \item $x3 = 249.9999$
    \item $x4 = 2.2717\times 10^{-7}$
    \item $x5 = 1.8168\times 10^{-8}$
\end{itemize}

也就是说分别应该采购加工原料油VEG1 $1.0039\times 10^{-6}\,\text{t}$, 原料油VEG2 $199.9999\,\text{t}$, 原料油OIL1 $249.9999\,\text{t}$, 原料油OIL2 $2.2717\times 10^{-7}\,\text{t}$, 原料油OIL3 $1.8168\times 10^{-8}\,\text{t}$。

\end{document}
